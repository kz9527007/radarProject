\documentclass[mathserif]{beamer}
\usepackage{graphicx}
\usecolortheme{dove}

\title{Moving Target Indicators}
\author{Swrangsar Basumatary (09d07040) \\ Chakradhar Thallapaka (09007046)}
\institute{Department of Electrical Engineering \\ IIT Bombay, Powai}
\date{April 23, 2014}

    
\begin{document}
    \frame{\titlepage}
    
    \begin{frame}{LEDs vs Laser Diodes for short-range communication}
\pause
\textbf{Why use LEDs when Laser Diodes are faster?}
            \pause For broadband short-range optical fiber communications, like
LANs and Fiber-in-the-Home networks, LEDs are:
            \pause
            \begin{itemize}
                \item cheaper
                \item safer for the human eyes
                \item less sensitive to temperature variations
                \item and more durable
            \end{itemize}
  
        \pause
        \textbf{Disadvantage of using LEDs}
            \begin{itemize}
                \pause \item The problem with LED is \pause \emph{low modulation rate!}\\
                \pause \item While laser diodes have reached to tens of Gbps, \pause
                commercial DH-LED (double heterostructure) is still limited at 100 Mbps.
            \end{itemize}
        
    \end{frame}
    
    \begin{frame}{Efforts that were not successful}
        \pause
        Efforts have been made to get upto 500 Mbps for conventional LEDs using
        \begin{itemize}
            \pause \item multilevel Pulse Amplitude Modulation (M-PAM)
            \pause \item and discrete multitone modulation (DMT) \\~\\
        \end{itemize}
        
        \pause But these techniques are \emph{highly complex} compared to the simple on-off keying (OOK) direct modulation scheme.
    \end{frame}
    
    
    
 
    
    \begin{frame}{References}
        \pause
        \begin{itemize}
                 \item Merrill I. Skolnik, \emph{``Introduction to Radar Systems''}, McGraw-Hill, 2001.
        \end{itemize}
    \end{frame}
    
    
\end{document}
